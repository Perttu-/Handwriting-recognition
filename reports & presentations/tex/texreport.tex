\documentclass{article}
\usepackage[margin=2cm]{geometry}
\usepackage{enumitem}
\usepackage{url}
\def\code#1{\texttt{#1}}

\title{%
    Handwriting Recognition \\
    \large Pre-processing and Layout Analysis}

\author{Perttu Pitk{\"a}nen}
\newlist{abbrv}{itemize}{1}
\setlist[abbrv,1]{label=,labelwidth=1in,align=parleft,itemsep=0.1\baselineskip,leftmargin=!}

\begin{document}

   \maketitle

   \newpage
   \addcontentsline{toc}{section}{Abstract}
   \section*{Abstract}
    Abstract text

   \newpage
   \addcontentsline{toc}{section}{Table of Contents}
   \tableofcontents


   \newpage
   \addcontentsline{toc}{section}{List of Abbreviations and Symbols}
   \section*{List of Abbreviations and Symbols}

   \begin{abbrv}
    \item[OCR] Optical Character Recognition
    \item[HWR] Handwriting Recognition
    \item[HOG] Histogram of Ordered Gradients
    \item[k] Number of selected neighbouring elements in k-nearest neighbors algorithm
   \end{abbrv}

   \newpage
   \section{Introduction}
    Optical Character Recognition (OCR) is the process of analyzing a image input of  text and recognizing and extracting the characters to digital from. More specific case of optical character recognition process is the task of handwriting recognition (HWR) which concentrates on analyzing human hand written characters instead of printed characters. The unpredictable nature of human handwriting can make the task more challenging ascompared to printed text. Most HWR systems can be divided into two recognition approaches: \textit{online handwriting recognition} and \textit{offline handwriting recognition}. Offline handwriting recognition means analyzing an existing static image for handwritten text. Online handwriting recognition, on the other hand, is about analyzing the handwritten text on input including strokes and their order for example in touch screen appliances such as smartphones and tablet PCs.

    Handwriting recognition can be applied to many practical uses such as document digitizaton or novel human-computer interaction method. Handwriting has remained popular as a way to take notes and transfer information, even though increased popularity and technological advancements of handheld digital devices have made digital information saving increasingly convinient. The process of handwriting recognition is still undergoing development.

    For reliable results the handwritten input image must be processed appropriately. This includes pre-processing and layout analysis of which aim to enhance the image quality for later processing and find the different bodies of text. There is no general consensus of which methods or algorithms give the best results. This research will experiment with different methods to gain insight on each method or algorithms strengths and weaknesses.

  \newpage
  \section{Background}
    Advacements with personal computers has diversified the methods to store and display textual information which has brought up new challenges and problems considering the transformation between traditional information and digital data. One of these challenges is the process of digitizing written text to computer readable and editable form.

    Textual information can be in diverse forms and styles. These styles include machine printed text and handwritten text. Different approaches must be used when digitizing aforementioned styles of text.

    Plenty of research has been conducted and several systems have been implemented for the purpose of optical character recognition and handwriting recognition. These systems can have drastically different approaches for processing the data, even if the data is similar.

  \subsection{Optical Character Recognition}
    Typical OCR systems consist of three phases:
    \begin{enumerate}
      \item{Pre-processing}
      \item{Layout Analysis}
      \item{Feature Extraction}
      \item{Classification}
    \end{enumerate}


    In image pre-processing stage the quality of image is enhanced and the area of interest is located. Additionally layout analysis phase can be included into the pre-processing stage. The feature extraction stage captures the distinctive characteristics of the digitized characters for recognition. Lastly in during the classification stage the feature vectors are processed to identify the characters and words. Each of these stages reduce the amount of information to be processed at a later step. \cite{Cheriet2007}
    [graph, diagram or smth. pls]

      \subsubsection{Pre-processing}
        At pre-processing stage the image is enhanced by applying varying filters, transforms and segmentation to it. For text recognition it is important to reduce noise from the image. This can be done with appropriate filter e.g. Wiener filter.

        Most of OCR softwares require the image to be binarized before analysis. It is important for later stages of recognition process that the binarized image contains as little noise and irrelevant objects as possible.  These irrelevant objects can be caused by for example uneven lighting, paper texture or other non-textual objects such as drawings. Binarization method should be chosen carefully as the input image's paper texture can vary a lot.

        Additionally, other ways to enhance the image before analysis have been experimented. Pesch et.al. discussed how contrast normalization, slant correction and size normalization can be applied to improve the handwriting recognition results. \cite{Pesch2012}

      \subsubsection{Layout Analysis}
        Layout analysis is the process to find where the actual text is located and what kind of textual blocks the image contains. These textual elements can contain titles, columns, captions consisting of text lines and words. Handwritten text is more likely to contain full page width single column text compared to printed text where more complex layouts are found more often. However handwritten text is more unpredictable compared to printed characters as the handwritten words can often overlap or have varying lineskews between lines.

      \subsubsection{Feature Extraction}
        To differentiate between words or individual characters some features must be extracted. These features can then be later used with classification stage to classify the input text.

        Several experiments have been conducted for appropriate features. For instance raw intensity values of selected component can be used. More sophisticated features include histogram of ordered gradients (HOG) \cite{Dalal2005}. S. Dalal et.al. have discussed other feature extraction methods such as horizontal and vertical projection histograms, parameters of polynominals i.e. curve fitting and topological features such as loops, end points, dots, and junctions. \cite{Dalal}

      \subsubsection{Classification}
        Lastly in the recognition process is the classification phase. The goal of classfication is to find the class in which the new input will be assigned and by doing that finding which is the most likely meaning of each particular component. In this case inputs are features extracted from word or character and classes are the corresponding words or characters respectively.

        Often during classification machine learning approaches are used. Machine learning algorithms look for repeating patterns in feature space and makes decisions and predictions according to those patterns. Common for machine learning algorithms is that they require some preliminary data to be processed in order to make later classification more robust. Machine learning algorithms that can be applied to text recognition include:
        \begin{itemize}
          \item Artificial neural networks
          \item K-Nearest Neighbor
          \item Hidden Markov model
          \item Support vector machine
          \item Recurrent neural networks
          \item Deep feedforward neural networks
          \item Decision tree learning
          \item Random forests
        \end{itemize}
          Simple example for machine learning is the k-Nearest Neighbor algorithm. The algorithm searches for the closest match of test data in the feature space. The previous training data is distributed in the feature space and classified accordingly. Specified amount of the nearest neighbors of the new node are counted and compared. The class has the most representation within these points is the class of the new node. The k stands for the amount of neighboring data points that are compared to the test data, and it should be declared as an odd number to prevent a tie from happening between two classes. The feature space can be constructed from feature vectors aquired in the previous phase. In general all machine learning algorithms work better when there are more feature dimensions, but this will result in a slower execution time.[----sources??----] [image also pls]

    \subsection{State of the Art}
      The subject of optical character recognition and handwriting recognition are widely researched subjects and well-functioning as well as feature rich software already exist. Many of these softwares utilize machine learning and neural networks to get satisfactory results especially with handwriting recognition. Many of the best OCR softwares are proprietary, thus making them unable for free research and analysis. Such software are for example Evernote which has an inbuilt OCR engine for searching text from pictures \cite{Kelly} and Abbyy FineReader software made especially for OCR \cite{ABBYY}. Examples of open-source OCR software are Tesseract\cite{Smith2007a}, OCRopus\cite{Breuel2007}, Ocrad\cite{FreeSoftwareFoundation2016} and CuneinForm\cite{CognitiveTechnologies2016}. These pieces OCR software are designed to process printed text and are not capable of handwriting recognition by default. Although for example Tesseract can be trained to detect any kind of text.\cite{Smith2007a} Additionally, handwriting recognition algorithms developed by Jürgen Schmidhuber's research group at the Swiss AI Lab IDSIA have won several international handwriting competitions. These algorithms utilize neural networks and deep learning. \cite{Angelica}.


  \newpage
  \section{Studied Methods (better title?)}
    The goal of this research is to gain insight on different methods that can be applied to pre-processing and layout analysis. Different methods were experimented with and the results were compared to merit and de-merit these methods and algorithms. Studied methods were aquired from various sources such as scientific papers or websites.
    \subsection{Methods for Pre-Processing}
      As mentioned before pre-processing requires various filters and transforms to enhance image quality. More specifically aiming the resulting image to contain only textual elements and as little noise as possible. For this research experimented methods are as follows:

        \subsubsection{Noise Removal}
          Noise in image can affect negatively to the rest of the recognition process. For that reason it is important to remove as much noise as possible without losing any textual information. For this purpose adaptive Wiener filter was chosen. Prior to using this filter image is converted to grayscale. Adaptive Wieners main advantage is that it can apply varying amounts of filtering to areas having different variations thus preserving textual information better than linear filtering.\cite{TheMathWorksWiener}
        \subsubsection{Contrast Enhancement}
          To make the text more prominent the contrast of the image can be enhanced. For this purpose
        \subsubsection{Binarization}
        \subsubsection{Component Property Analysis}
        \subsubsection{Stroke Width Analysis}


    \subsection{Methods for Layout Analysis}
      To find information about tex location within the image layout analysis methods should be applied. The goal is to exclude irrelevant components that might remain after pre-proessing and extract only the textual objects from the input image with information about the relative location within the document. Following methods were used in experiments:
      \begin{description}
        \item [Bounding Box Expansion Method] [thorough explanations and images and stuff pls]
        \item [Run Length Smearing Algorithm (RLSA)]
        \item [Block Based Hough Transform Mapping]
      \end{description}

  \newpage
  \section{Implementation}
    The following chapter describes in detail the used methods, algorithms and software architecture. All the software was implemented with MATLAB using the Image Processing Toolbox for image acquisition and pre-processing as well as layout analysis.\cite{TheMathWorksa} The MATLABs inbuilt functions and libraries also provides methods for graphical user interface implementation and data visualization. Code excerpts in this chapter use MATLAB syntax if not stated otherwise.

    \subsection{Pre-processing Implementation}
      \begin{description}
        \item [Image acquisition] Matlab can process jpg, png and gif images to name a few.
        \item [Conversion to grayscale] Matlab function \code{rgb2gray(I)} eliminates hue and saturation information and keeps lumination. The image must be rgb color model before conversion.
        \item [Noise removal] This is done with n x m adaptive Wiener filter provided in MATLAB image processing toolbox. The filter size must be defined beforehand. In this case the filter size will be adjusted according to input data and test results.
        \item [Binarization] Sauvola algorithm as it has been developed specifically for document image binarization. The algorithm is enhanced version of Nilback algorithm. \cite{Sauvola2000}
        Following  constraints must be defined prior to binarization procedure: The dimensions of MxN neighborhood used to determine the adaptive threshold for that area. Threshold k ``sensitivity'' is user defined parameter which is used the Nilback binarization algorithm to define how the algorithm handles noise. Smaller values remove more noise but result in more fractioned results. There is no consensus for choosing the aforementioned values so they should be chosen case by case basis.
      \end{description}

      \subsection{Layout Analysis Implementation}
        \begin{description}
          \item [Object removal based on stroke width variation]
          \item [Louloudis method for text line segmentation]
        \end{description}
  \newpage
  \section{Evaluation}
  evltn [lots of graphs and tables, por favor]

  \newpage
  \bibliographystyle{abbrv}
  \bibliography{hwr_bibliography}



\end{document}
